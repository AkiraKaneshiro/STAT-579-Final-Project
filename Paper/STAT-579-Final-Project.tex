\documentclass[11pt]{article}
\usepackage[margin=1.2in]{geometry}
\usepackage{graphicx}
\usepackage{fancyhdr}
\usepackage{nopageno}%removes number on titlepage
\usepackage[parfill]{parskip}
\usepackage{subfig}
\usepackage{multirow}

\begin{document}
% Cover page
\author{Chelsey Legacy, Lindong Zhou, Evan Pete Walsh\footnote{Statistics graduate students at Iowa State University of Science and Technology}}
\title{STAT 579 Final Project}
\maketitle

\begin{abstract}
Analysis of the Yelp Academic Dataset.
\end{abstract}

\newpage

\tableofcontents

\newpage

\pagenumbering{arabic}%starts numbering at "1"
\pagestyle{fancy}
\fancyhead[L]{Iowa State University}
\rhead{\thepage}
\rfoot{\today}
\cfoot{}%removes default page number at bottom center


\section{Introduction}

Lindong


\section{Data}

The Yelp Academic Dataset\footnote{$https://www.yelp.com/academic\_dataset$} provides data enthusiants with the exciting opportunity to explore an incredible collection of information regarding the characteristics and quality of hundreds of businesses across the United States, Canada, and the UK. Specifically, the data includes details and reviews on 250 of the closest businesses to 30 large universities, including the Arizona State, UNLV, the University of Edinburgh, the University of Wisconsin, and the University of Waterloo, to name a few. The raw data is in json format and contains five different types of json objects: \textbf{Business}, \textbf{Review}, \textbf{User}, \textbf{Check-in}, and \textbf{Tip}.

Each \textbf{Review} object represents an individual user-based review of a particular business. The unique encrypted business ID is given along with the date of the review, the number of stars (out of 5) that were awarded, the number and type of votes that the review received, and an optional text description provided by the user. \textbf{User} objects are unique to every person that has an active Yelp account. Each user has a name, a unique encrypted user ID, the number of votes they have cast, the average number of stars they have given, and the date they signed up for Yelp, among other things. A \textbf{Check-in} object represents the count and time of all the registered check-ins for a particular business, and \textbf{Tip} objects represent a tip given by user for a particular business. Tips include the user's ID, the business's ID, the date, and the message that the user gave. While these objects all provide a rich source of information, for the scope of this paper we will primarily be examing the \textbf{Business} objects.

\textbf{Business} objects are unique to business ID's, and include the following information:
\begin{itemize}
	\item the name of the business,
	\item the name of neighborhood in which the business is located,
	\item the city in which the business is located,
	\item the full address of the business,
	\item the exact latidude and longitude coordinates of the business,
	\item the average number of stars awarded to the business,
	\item the number of reviews received by the business,
	\item whether or not the business is still open,
	\item the hours that the business is open,
	\item the categories that the business falls under,
	\item and a number of different attributes which mostly concern restaurants and bars, such as whether or not smoking is allowed and the price range of the food.
\end{itemize}

To work with the data, we converted the set of all \textbf{Business} objects to a csv file in which the columns are variable names representing each aspect of a \textbf{Business} object, and each row corresponds to a unique business. Each different type of attribute was converted to its own character, numerical, or logical variable depending on what was appropriate. For example, the attribute \textbf{smoking} was converted to a logical variable where the value is ``true" if the smoking is allowed at the establishment, and ``false" if it is not allowed, while the attribute \textbf{price range} was converted to a numerical variable which ranges from 1 to 4.

\section{Restaurants}

Most of the attributes from the \textbf{Business} objects were meant to describe food and drink-serving establishments, making restaurants the most interesting type of business to explore.

Most of the data about restaurants in the Yelp Academic Dataset are concentrated around the following cities:
\begin{itemize}
	\item Madison, WI, 
	\item Las Vegas, NV,
	\item Pheonix, AZ,
	\item Toronto, Canada,
	\item and Edinburgh, Scotland.
\end{itemize}
Although the choise of cities is limited, the wealth of information coming from each city is staggering. For example, 

\section{Bars}

Chelsey

\section{Hotels}

Lindong

\section{Fitness}

Lindong


\section{Conclusion}


Chelsey




\end{document}